%\section{Vector Green Function}
We will calculate the Green function for a massless vector gauge boson. The field $A_\mu$ is described by the equation
\begin{equation}\label{eq:Fpde}
D_\mu F_{\mu\nu} = J_\nu,
\end{equation}
where 
\begin{equation}\label{eq:Fexp}
F_{\mu\nu}=D_\mu A_\nu - D_\nu A_\mu
\end{equation}
and the source of the field, $J_\nu$, is a conserved current:
\begin{equation}\label{eq:Jcons}
\mathrm{div} J = D_\nu J^\nu = \frac{1}{\sqrtg} \partial_\nu (\sqrtg J^\nu) = 0.
\end{equation}

Inserting eq.~(\ref{eq:Fexp}) into eq.~(\ref{eq:Fpde}) gives
$$J^\nu = D^\mu \left( D_\mu A_\nu - D_\nu A_\mu \right) = D^\mu D_\mu A_\nu - D^\mu D_\nu A_\mu.$$
The last term can be rewritten:
$$D^\mu D_\nu A_\mu = D_\nu D^\mu A_\mu + R^\rho{}_{\mu\nu}{}^\mu A_\rho = D_\nu (D \cdot A) + R^\rho{}_\nu A_\rho.$$
Inserting $R^\rho{}_\nu = -d\, \delta_\nu^\rho$ thus gives us
$$J^\nu = \Box A_\nu - D_\nu (D \cdot A) + d A_\nu.$$
Now, since eq.~(\ref{eq:Fpde}) only contains $F_{\mu\nu}$, which is invariant under the gauge transformation $A_\mu \to A_\mu + D_\mu \Lambda$, we can choose our gauge $\Lambda$ so that $D \cdot A = D^\mu A_\mu = 0$ (Lorentz gauge). This gives us our final equation for $A_\mu$:
\begin{equation}\label{eq:Apde}
\Box A_\mu + d A_\mu = J_\mu.
\end{equation}

The solution of eq.~(\ref{eq:Apde}) is given by
$$A_\mu(x) = \int G_{\mu\nu'}(x,x') J^{\nu'}(x')\, \ud V(x'),$$
where $G_{\mu\nu'}(x,x')$ is the Green function. This is a two-point function which is a \emph{one-index tensor at each of the points $x$ and $x'$.} Primed indices (like $\nu'$ above) of course refer to $x'$ while unprimed ($\mu$) indices refer to $x$. The Green function should satisfy
$$J_\mu(x) = \int (\Box^{(x)} + d)G_{\mu\nu'}(x,x') J^{\nu'}(x')\, \ud V(x').$$
It is then natural to suppose that one should have \footnote{$g_{\mu\nu'}(x,x')$ is the \emph{parallel propagator}; for $x=x'$ it is just $g_{\mu\nu}(x)$.}
$$(\Box^{(x)} + d)G_{\mu\nu'}(x,x') = g_{\mu\nu'}(x,x') \delta_\AdS(x,x'),$$
but this is not necessarily the case. The right hand side could differ from this by a derivative $\partial_\mu \Lambda$, since
\begin{equation}\label{eq:equivgnd}
\begin{split}
\int \partial_\nu \Lambda\, J^\nu\, \ud V &= \int \partial_\nu \Lambda\, J^\nu \sqrtg\, \ud^{d+1}x \\
&= -\int \Lambda\, \partial_\nu \left(\sqrtg J^\nu\right) \ud^{d+1}x \\
&= -\int \Lambda\, \frac{1}{\sqrtg} \partial_\nu \left( \sqrtg J^\nu \right) \sqrtg\, \ud^{d+1}x \\
&= -\int \Lambda\, \mathrm{div} J\, \ud V = 0,
\end{split}
\end{equation}
where we have assumed $J$ to vanish at infinity so that we get no boundary terms.
For this reason, we introduce an equivalence relation by
$$A_\nu \sim B_\nu$$
if for some $\Lambda$
$$A_\nu - B_\nu = \partial_\nu \Lambda$$
so that
$$\int A_\nu\, J^\nu\, \ud V = \int B_\nu\, J^\nu\, \ud V.$$

We should thus have
\begin{equation}\label{eq:Gpde}
(\Box^{(x)} + d)G_{\mu\nu'}(x,x') \sim g_{\mu\nu'}(x,x') \delta_\AdS(x,x').
\end{equation}

The AdS invariance of $G_{\mu\nu}$ forces it to be a linear combination of
$D_\mu D_{\nu'} u$ and $D_\mu u\,  D_{\nu'} u$,
$$G_{\mu\nu'} = F(u) D_\mu D_{\nu'} u + S(u) D_\mu u\, D_{\nu'}u.$$
It is however easily shown that
$$F(u) D_\mu D_{\nu'} u \sim - F'(u) D_\mu u\, D_{\nu'}u,$$
so we only have to consider one kind of term. We choose to work with $D_\mu D_{\nu'} u$ and thus make the ansatz
$$G_{\mu\nu'} = F(u) D_\mu D_{\nu'} u.$$

Using the identity
$$\Box \left( A B \right) = \Box A\, B + 2 D^\rho A\, D_\rho B + A\, \Box B$$
we get
\begin{equation}\label{eq:boxG}
\Box^{(x)}G_{\mu\nu'} = \Box^{(x)} F(u)\, D_\mu D_{\nu'} u + 2 D^\rho F(u)\, D_\rho D_\mu D_{\nu'} u + F(u)\, \Box^{(x)} D_\mu D_{\nu'} u.
\end{equation}

Here, using the properties of $u$ from \todo{Appendix ???} we have
\begin{equation*}
\begin{split}
\Box^{(x)} F(u) &= D^\rho D_\rho F(u) = D^\rho \left[ F'(u) D_\rho u \right] \\
&= F''(u) D^\rho u\, D_\rho u + F'(u) D^\rho D_\rho u \\
&= u (2+u) F''(u) + (d+1)(1+u) F'(u),
\end{split}
\end{equation*}
\begin{equation*}
\begin{split}
D^\rho F(u)\, D_\rho D_\mu D_{\nu'} u &= F'(u)\, D^\rho u\, g_{\rho\mu} D_{\nu'} u \\
&= F'(u)\, D_\mu u\, D_{\nu'} u \\
&\sim -F(u) D_\mu D_{\nu'} u
\end{split}
\end{equation*}
and
$$\Box^{(x)} D_\mu D_{\nu'} u = D_\mu D_{\nu'} u.$$
Inserting this into (\ref{eq:boxG}) gives
$$\Box^{(x)}G_{\mu\nu'} \sim \Big( u(2+u) F''(u) + (d+1)(1+u) F'(u) - F(u) \Big)\, D_\mu D_{\nu'} u$$
which makes (\ref{eq:Gpde}) become
\begin{multline*}
\Big( u(2+u) F''(u) + (d+1)(1+u) F'(u) +(d-1) F(u) \Big)\, D_\mu D_{\nu'} u \\
\sim g_{\mu\nu'}(x,x') \delta_\AdS(x,x').
\end{multline*}

For $x \neq x'$, we thus (???really???) shall have
$$u(2+u) F''(u) + (d+1)(1+u) F'(u) + (d-1) F(u) = 0.$$
Comparing this with (\ref{eq:Hode}) directly gives us the solution; it is given by setting $m^2 = -(d-1)$ (i.e.\ $\nu=d/2-1$, $\Delta = d-1$) in (\ref{eq:scalargreen}):
\begin{equation*}
\begin{split}
F(u) &= C \frac{1}{(u+1)^\Delta} {}_2F_1 \left( \frac{\Delta}{2}, \frac{\Delta+1}{2}; \nu+1; \frac{1}{(u+1)^2} \right) \\
&= C \frac{1}{(u+1)^{d-1}} {}_2F_1 \left( \frac{d-1}{2}, \frac{d}{2}; \frac{d}{2}; \frac{1}{(u+1)^2} \right) \\
&= C [u(u+2)]^{-(d-1)/2}
\end{split}
\end{equation*}
for some constant $C$, not necessarily having the same value as in (\ref{eq:scalargreen}).

\todo{...}
$$G_{\mu\nu'} = C [u(u+2)]^{-\frac{d-1}{2}}\, D_\mu D_{\nu'} u$$
\todo{Insert expression for $D_\mu D_{\nu'} u$ and check behavior for $u \sim 0$ (and $u \to \infty$)}