\subsection{Parametrisation \& Coordinates}
\todo{Can really have this text... :-)}
There are several ways to parametrise AdS~spaces. Some are intuitive, but difficult to work with. Others are non-intuitive, but easy to work with.

\subsubsection{Polar coordinates}
Let's set $r = |\vec{X}|$ and $\vec{n} = \vec{X}/r$. Then $\vec{n} \in S^{d-1}$, so $\vec{n}^2 = 1$ and $\vec{n} \cdot \ud\vec{n} = 0$. From this we deduce that
$$\ud \vec{X}^2 = (\ud r\:\vec{n} + r\:\ud\vec{n})^2 = \ud r^2 + r^2\:\ud\vec{n}^2.$$

The time coordinates then satisfy $T_1^2 + T_2^2 = \vec{X}^2 + R^2 = r^2 + R^2$, i.e. they describe a circle, and are most naturally parametrised by
\begin{align*}
T_1 = \sqrt{r^2+R^2}\cos{t}\\
T_2 = \sqrt{r^2+R^2}\sin{t}
\end{align*}
for $-\infty<t<+\infty$ (remember that we are considering the covering space).
\\

The infinitesimals of $T_1$ and $T_2$ becomes
\begin{align*}
\ud T_1 &= \frac{r\:\ud r}{\sqrt{r^2+R^2}}\cos{t} - \sqrt{r^2+R^2}\:\ud t\:\sin{t}\\
\ud T_2 &= \frac{r\:\ud r}{\sqrt{r^2+R^2}}\sin{t} + \sqrt{r^2+R^2}\:\ud t\:\cos{t}
\end{align*}

Thus,
$$\ud T_1^2 + \ud T_2^2 = \frac{r^2\:\ud r^2}{r^2+R^2} + (r^2+R^2) \ud t^2$$

Now we can calculate the metric for the AdS space in terms of the polar coordinates:

\begin{equation}\begin{split}
\ud s^2 &= \ud \vec{X}^2 - \ud T_1^2 - \ud T_2^2 \\
     &= (\ud r^2 + r^2\:\ud\vec{n}^2) - \left(\frac{r^2\:\ud r^2}{r^2+R^2} + (r^2+R^2) \ud t^2\right)\\
     &= \left(1 - \frac{r^2}{r^2 + R^2}\right)\:\ud r^2 + r^2\:\ud\vec{n}^2 - (r^2+R^2)\:\ud t^2\\
     &= \frac{R^2}{r^2+R^2}\:\ud r^2 + r^2\:\ud\vec{n}^2 - (r^2+R^2)\:\ud t^2
\end{split}\end{equation}

This parametrisation can be varied by setting $r$ to be a function of some parameter $\rho$ to get nice properties of the metric. For example:
\todo{Formulering}
\begin{itemize}
\item
Setting $r = {2 R \rho}/({1 - \rho^2}), \quad 0 \leq \rho < 1 \quad$ gives
$$\ud s^2 = R^2\left(\frac{4}{(1-\rho^2)^2}(\ud\rho^2+\rho^2\:\ud\vec{n}^2) - \left(\frac{1+\rho^2}{1-\rho^2}\right)^2\:\ud t^2\right),$$
where the {\em Poincar\'e metric} for $d-1$ dimensions can be recognised in $\rho$ and $\vec{n}$ (for fixed $t$).
\item
Setting $r = R \tan{\rho}, \quad 0 \leq \rho < 1 \quad$ gives
$$\ud s^2 = \frac{R^2}{\cos^2{\rho}}(\ud\rho^2 + \sin^2{\rho}\:\ud\vec{n}^2 - \ud t^2)$$
in which a light signal emitted from $r=\rho=0$ propagates like $\rho = t$. 
\todo{Emphasize this fact! Light travels to the boundary in a finite time!}
\end{itemize}


\subsubsection{Standard parametrisation}

%\todo{Change variable names?}
This is one of those ``non-intuitive, but easy to work with''-parametrisations, since it's quite like the Euclidean or Minkowskian metrics. (!!!Metrics here, since plural?!!!).
However, it only covers half of the non-covered AdS space, but that will anyway be enough for us.

\todo{Ha vektorbeteckning p� Y? $\vec{Y}$?}
Start by setting $X_{+} = X_1 + T_1$, \, $X_{-} = X_1 - T_1$ and $Y = (X_2, X_3, \dots, X_d, T_2)$ (signature $(+,\dots,+,-)$). Then (\ref{eq:AdS}) becomes
$$X_{+}X_{-} + Y^2 = -1$$ 
from which we can solve for $X_{-}$, getting
$$X_{-} = -\frac{Y^2 + 1}{X_{+}}.$$
The differential of $X_{-}$ is
$$\ud X_{-}=-\frac{2Y\cdot \ud Y}{X_{+}}+\frac{1+Y^2}{X_{+}^2}\ud X_{+}.$$
Then the metric becomes
$$\ud s^2=\ud X_{+}\:\ud X_{-} + \ud Y^2 = -2Y\cdot \ud Y\frac{\ud X_{+}}{X_{+}}+(1+Y^2)\frac{\ud X_{+}^2}{X_{+}^2} + \ud Y^2.$$
Now, setting $Y=X_{+} Z$, we get
$$\ud Y=\ud X_{+} Z + X_{+} \ud Z$$
giving
$$Y\cdot \ud Y = X_{+} \: \ud X_{+} \: Z^2 + X_{+}^2 Z\cdot \ud Z$$
$$\ud Y^2 = \ud X_{+}^2 Z^2 + 2X_{+} \: \ud X_{+} \: Z\cdot \ud Z + X_{+}^2 \ud Z^2$$
Inserting this into (!!!the metric!!!) gives (after simplification)
$$\ud s^2 = \frac{\ud X_{+}^2}{X_{+}^2} + X_{+}^2 \ud Z^2.$$
Now, we just set $\rho = 1/X_{+}$ and get
$$\ud s^2 = \frac{\ud\rho^2 + \ud Z^2}{\rho^2}$$
Here, however, we will rename the variables by $\rho \equiv x_0$ and $Z \equiv \vec{x}$. We also introduce the symbol $x = (x_0, \vec{x})$:
$$\ud s^2 = \frac{{\ud x_0}^2 + {\ud\vec{x}}^2}{{x_0}^2} = \frac{\ud x^2}{x_0^2}.$$

\todo{L�gga in $x \cdot y \equiv x_0 y_0 + \vec x \cdot \vec y$ ???}

We will continue to use the variables $(X_+, X_-, Y)$ when we need a linkage between $(\vec{X}, T_1, T_2)$ and $(x_0, \vec{x})$.