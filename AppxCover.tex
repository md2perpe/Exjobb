%\appendix
\chapter{Covering Space}
\section{Mathematical Definition}
\newtheorem{dfn}{Definition}
\begin{dfn}
A continuous map $F:A \subset R^n \to R^n$ is a {\em homeomorphism} onto $F(A)$ if $F$ is one-to-one and the inverse $F^{-1}:F(A) \subset R^n \to R^n$ is continuous. In this case $A$ and $F(A)$ are {\em homeomorphic sets}.
\end{dfn}
\begin{dfn}
If $X$ and $Y$ are two spaces, we say that $\pi: X \to Y$ is a {\em covering map} if
  \begin{enumerate}
	\item $\pi$ is continuous and $\pi(X)=Y$.
	\item Each point $p \in Y$ has a neighborhood $U$ in $Y$ (to be called a {\em distinguished neighborhood of $p$} such that $$\pi^{-1}(U) = \bigcup_{\alpha} V_{\alpha},$$
where the $V_{\alpha}$'s are pairwise disjoint open sets such that the restriction of $\pi$ to $V_{\alpha}$ is a homeomorphism of $V_{\alpha}$ onto $U$.
  \end{enumerate}
$X$ is then called a {\em covering space} of $Y$.
\end{dfn}
\section{What it Means in Practice}
\todo{Change name of section?}

